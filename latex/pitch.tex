\documentclass[12pt,svgnames]{article}
\usepackage{tikz}
\usepackage[frenchb]{babel}
\usepackage[utf8]{inputenc}
\usepackage[T1]{fontenc}
\usepackage[margin=2.5cm]{geometry}
\usepackage[parfill]{parskip}
\usepackage{xcolor}
\usepackage{titlesec}
\usepackage[Lenny]{fncychap}
\titleformat{\section}
  {\normalfont\sffamily\Large\bfseries\color{teal}}
  {\thesection}{1em}{}
\begin{document}
\thispagestyle{empty}
{\fontfamily{phv}\selectfont
    \section*{Ce que nous faisons}
    Nous intervenons sur des projets naissants ou existants pour lesquels se posent sur la durée les questions de l'énergie, de la motivation et de la satisfaction de l'équipe et des utilisateurs.\\

    Dans ces projets on peut retrouver certaines dynamiques : des problèmes qui consument de l'énergie et d'autres qui en donnent, le partage ou la segmentation des enjeux et des obstacles, la circulation des informations dans les échanges\ldots{} Ces dynamiques, quand elles sont peu visibles, peuvent en amener d'autres, telles que l'éloignement de l'équipe de sa préoccupation initiale de rendre service aux utilisateurs, de l'insatisfaction, du turn-over\ldots{}\\

    Notre approche consiste à :
    \begin{itemize}
        \item s'intéresser d'abord au contexte dans lequel les pratiques sont apparues et se sont ancrées pour permettre de les améliorer 
        \item voir le développement logiciel moins comme une activité de fabrication que comme une activité de traduction d'idées, et aider le triangle \emph{Tech - Métier - Management} à produire des traductions cohérentes et créatrices de valeur
        \item considérer l'apprentissage (aussi bien technique que métier) comme nécessaire à l'amélioration et à la production de sens, et donc le replacer au centre de l'activité
    \end{itemize}
    \section*{Nos Outils (fondamentaux)}
    \begin{itemize}
        \item
            programmation orientée objet, fonctionnelle
        \item
            Test Driven Design, tests d'intégration, tests exploratoires
        \item
            eXtreme Programming, Lean, apprentissage
    \end{itemize}
    \section*{Nos Outils (en mots-clés)}
    \begin{itemize}
        \item
            Java, JavaScript, C, C\#, SQL, Python, Haskell, Elm, Lisp
        \item
            Junit, Nunit, Jest, Jasmine, Mocha, Sinon, Approval Tests, Characterization Tests, Cucumber, Fitnesse, Exploratory Testing, Domain Testing
        \item
            SQL, Oracle, Sqlite, Postgres
        \item
            Scrum, Lean, Event Storming, Coding Dojo, Code Kata, Mob Programming, Pair Programming, Event Storming, Wardley Maps, Appreciative Coaching, Open Space Technology, World Café, Liberating Structures, Non Violent Communication, Appreciative Inquiry, Clean Language, Systemic Modelling
        \item
            TDD/Legacy, BDD, Code Review, Writing Effective Use Cases, Agile Methodologies, Agile Coaching, Structured Feedback, Quality Management, Technical Leadership, Training From The Back of The Room
        \item
            Core Protocols, Logical Thinking Process, Influencer
    \end{itemize}
    \end{document}
